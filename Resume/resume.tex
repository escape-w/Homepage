%% Use the "normalphoto" option if you want a normal photo instead of cropped to a circle
% \documentclass[10pt,a4paper,normalphoto]{altacv}

\documentclass[10pt,a4paper,ragged2e,withhyper]{altacv}
%% AltaCV uses the fontawesome5 and packages.
%% See http://texdoc.net/pkg/fontawesome5 for full list of symbols.

% Change the page layout if you need to
\geometry{left=1.25cm,right=1.25cm,top=1.5cm,bottom=1.5cm,columnsep=1.2cm}

% The paracol package lets you typeset columns of text in parallel
\usepackage{paracol}

% Change the font if you want to, depending on whether
% you're using pdflatex or xelatex/lualatex
% WHEN COMPILING WITH XELATEX PLEASE USE
% xelatex -shell-escape -output-driver="xdvipdfmx -z 0" sample.tex
\ifxetexorluatex
  % If using xelatex or lualatex:
  \setmainfont{Roboto Slab}
  \setsansfont{Lato}
  \renewcommand{\familydefault}{\sfdefault}
\else
  % If using pdflatex:
  \usepackage[rm]{roboto}
  \usepackage[defaultsans]{lato}
  % \usepackage{sourcesanspro}
  \renewcommand{\familydefault}{\sfdefault}
\fi

% Change the colours if you want to
\definecolor{SlateGrey}{HTML}{2E2E2E}
\definecolor{LightGrey}{HTML}{666666}
\definecolor{DarkPastelRed}{HTML}{450808}
\definecolor{PastelRed}{HTML}{8F0D0D}
\definecolor{GoldenEarth}{HTML}{E7D192}
\colorlet{name}{black}
\colorlet{tagline}{PastelRed}
\colorlet{heading}{DarkPastelRed}
\colorlet{headingrule}{GoldenEarth}
\colorlet{subheading}{PastelRed}
\colorlet{accent}{PastelRed}
\colorlet{emphasis}{SlateGrey}
\colorlet{body}{LightGrey}

% Change some fonts, if necessary
\renewcommand{\namefont}{\Huge\rmfamily\bfseries}
\renewcommand{\personalinfofont}{\footnotesize}
\renewcommand{\cvsectionfont}{\LARGE\rmfamily\bfseries}
\renewcommand{\cvsubsectionfont}{\large\bfseries}


% Change the bullets for itemize and rating marker
% for \cvskill if you want to
\renewcommand{\cvItemMarker}{{\small\textbullet}}
\renewcommand{\cvRatingMarker}{\faCircle}
% ...and the markers for the date/location for \cvevent
% \renewcommand{\cvDateMarker}{\faCalendar*[regular]}
% \renewcommand{\cvLocationMarker}{\faMapMarker*}


% If your CV/résumé is in a language other than English,
% then you probably want to change these so that when you
% copy-paste from the PDF or run pdftotext, the location
% and date marker icons for \cvevent will paste as correct
% translations. For example Spanish:
% \renewcommand{\locationname}{Ubicación}
% \renewcommand{\datename}{Fecha}


%% Use (and optionally edit if necessary) this .tex if you
%% want to use an author-year reference style like APA(6)
%% for your publication list
% % When using APA6 if you need more author names to be listed
% because you're e.g. the 12th author, add apamaxprtauth=12
\usepackage[backend=biber,style=apa6,sorting=ydnt]{biblatex}
\defbibheading{pubtype}{\cvsubsection{#1}}
\renewcommand{\bibsetup}{\vspace*{-\baselineskip}}
\AtEveryBibitem{%
  \makebox[\bibhang][l]{\itemmarker}%
  \iffieldundef{doi}{}{\clearfield{url}}%
}
\setlength{\bibitemsep}{0.25\baselineskip}
\setlength{\bibhang}{1.25em}


%% Use (and optionally edit if necessary) this .tex if you
%% want an originally numerical reference style like IEEE
%% for your publication list
\usepackage[backend=biber,style=ieee,sorting=ydnt,defernumbers=true]{biblatex}
%% For removing numbering entirely when using a numeric style
\setlength{\bibhang}{1.25em}
\DeclareFieldFormat{labelnumberwidth}{\makebox[\bibhang][l]{\itemmarker}}
\setlength{\biblabelsep}{0pt}
\defbibheading{pubtype}{\cvsubsection{#1}}
\renewcommand{\bibsetup}{\vspace*{-\baselineskip}}
\AtEveryBibitem{%
  \iffieldundef{doi}{}{\clearfield{url}}%
}


%% sample.bib contains your publications
\addbibresource{sample.bib}

\begin{document}
\name{Devendra Sahu}
\tagline{Senior DevOps Engineer}
%% You can add multiple photos on the left or right
\photoR{2.8cm}{photo}
% \photoL{2.5cm}{Yacht_High,Suitcase_High}

\personalinfo{%
  % Not all of these are required!
  \email{75devendrasahu@gmail.com}
  \phone{+91-7566503852}
  % \mailaddress{Åddrésş, Street, 00000 Cóuntry}
  \location{Gurgaon, India}
  % \homepage{www.homepage.com}
  % \twitter{@twitterhandle}
  \linkedin{www.linkedin.com/in/thisisdevendrasahu}
  \github{https://github.com/escape-w}
  %\orcid{0000-0000-0000-0000}
  %% You can add your own arbitrary detail with
  %% \printinfo{symbol}{detail}[optional hyperlink prefix]
  % \printinfo{\faPaw}{Hey ho!}[https://example.com/]

  %% Or you can declare your own field with
  %% \NewInfoFiled{fieldname}{symbol}[optional hyperlink prefix] and use it:
  % \NewInfoField{gitlab}{\faGitlab}[https://gitlab.com/]
  % \gitlab{your_id}
  %%
  %% For services and platforms like Mastodon where there isn't a
  %% straightforward relation between the user ID/nickname and the hyperlink,
  %% you can use \printinfo directly e.g.
  % \printinfo{\faMastodon}{@username@instace}[https://instance.url/@username]
  %% But if you absolutely want to create new dedicated info fields for
  %% such platforms, then use \NewInfoField* with a star:
  % \NewInfoField*{mastodon}{\faMastodon}
  %% then you can use \mastodon, with TWO arguments where the 2nd argument is
  %% the full hyperlink.
  % \mastodon{@username@instance}{https://instance.url/@username}
}

\makecvheader
%% Depending on your tastes, you may want to make fonts of itemize environments slightly smaller
% \AtBeginEnvironment{itemize}{\small}

%% Set the left/right column width ratio to 6:4.
\columnratio{0.6}

% Start a 2-column paracol. Both the left and right columns will automatically
% break across pages if things get too long.
\begin{paracol}{2}
\cvsection{Experience}

\cvevent{Senior DevOps Engineer}{Nagarro}{Jan 2023 -- Ongoing}{Porto}
\begin{itemize}
\item Created reusable github workflows at org level.
\item Created ITSM flow for regular task by fully IaC terraform on AWS lambda
\item ITSM alerting by containerization in AWS Lambda by IaC
\item Python scripts and packages for automation ITSM task.
\item Github actions for onboarding users on Github.
\item Setup coding standards at department level by using pre-commits.
\item Cert manager deployment by Argo CD using helm
\item dockerize terraform and other applications.
\item Setup full IaC for application using terragrunt.
\item Develop yaml validation script to validate custom yamls
\end{itemize}

\divider

\cvevent{DevOps Engineer}{Nagarro}{Oct 2021 -- Jan 2023}{remote}
\begin{itemize}
\item Worked extensively on Azure Devops and created pipelines for IBM ACE, API Connect, Java pipelines
\item Automate day to day stuff for IBM ACE Integration and containerize them.
\item shell scripting for automation
\item Installation and other support related automation to reduce human efforts.
\end{itemize}

\divider

\cvevent{DevOps Engineer}{TCS}{Sep 2019 -- Sep 2021}{Noida}
\begin{itemize}
\item Setup and created shared libraries for jenkins in groovy. 
\item IaC to build AWS resources.
\item Proficient in writing IaC using Terraform to build the AWS and Azure Infra Use codedeploy pipelines for release for web apps Supported cicd pipelines
\item Worked on AIOps tool for auto resolve tickets in service now.
\item Worked with security tools like sonarqube, trivy
\end{itemize}

\divider

\cvevent{System Engineer}{TCS}{Dec 2016 -- Aug 2019}{Gandhinagar}
\begin{itemize}
\item Support and Maintain IBM MQ, automate day to day process by shell and powershell scripts 
\item LCM of IBM MQ with shell and ansible on 500 servers / Handle SSL certificate for application by python and ansible. 
\end{itemize}

\medskip

%\cvsection{A Day of My Life}

% Adapted from @Jake's answer from http://tex.stackexchange.com/a/82729/226
% \wheelchart{outer radius}{inner radius}{
% comma-separated list of value/text width/color/detail}
%\wheelchart{1.5cm}{0.5cm}{%
%  6/8em/accent!30/{Sleep,\\beautiful sleep},
%  3/8em/accent!40/Hopeful novelist by night,
%  8/8em/accent!60/Daytime job,
%  2/10em/accent/Sports and relaxation,
%  5/6em/accent!20/Spending time with family
%}

% use ONLY \newpage if you want to force a page break for
% ONLY the current column
%\newpage

\cvsection{Publications}

%% Specify your last name(s) and first name(s) as given in the .bib to automatically bold your own name in the publications list.
%% One caveat: You need to write \bibnamedelima where there's a space in your name for this to work properly; or write \bibnamedelimi if you use initials in the .bib
%% You can specify multiple names, especially if you have changed your name or if you need to highlight multiple authors.
\mynames{Lim/Lian\bibnamedelima Tze,
  Wong/Lian\bibnamedelima Tze,
  Lim/Tracy,
  Lim/L.\bibnamedelimi T.}
%% MAKE SURE THERE IS NO SPACE AFTER THE FINAL NAME IN YOUR \mynames LIST

\nocite{*}

\printbibliography[heading=pubtype,title={\printinfo{\faFile*[regular]}{Articles}},type=article]

\divider

%\printbibliography[heading=pubtype,title={\printinfo{\faUsers}{Conference Proceedings}},type=inproceedings]

%% Switch to the right column. This will now automatically move to the second
%% page if the content is too long.
\switchcolumn

\cvsection{CORE PHILOSOPHY}

\begin{quote}
``Striving for innovation, growth, and excellence in every aspect of life.''
\end{quote}

\cvsection{Skills}
\cvtag{Kubernetes}
\cvtag{AWS}
\cvtag{Terraform}
\cvtag{Ansible} 
\cvtag{Docker}
\cvtag{Github Actions}
\cvtag{jenkins}
\cvtag{Python}
\cvtag{GIT}
\cvtag{CI/CD pipelines}
\cvtag{SDLC}
\cvtag{Azure DevOps}
\cvtag{problem-solving}
\cvtag{Containerisation}
\cvtag{JIRA}
\cvtag{Automation}
\cvtag{Prometheus}
\cvtag{Azure} 
\cvtag{ITSM/ITIL4}
\cvtag{SRE}
\cvtag{bash}
\cvtag{scripting}

\cvsection{Certifications}

\cvachievement{\faTrophy}{Certified Kubernetes Administrator}{(04/2023 - 04/2026)}

\divider

\cvachievement{\faTrophy}{Microsoft Certified: DevOps Engineer Expert}{AZ-400}

\divider

\cvachievement{\faTrophy}{HashiCorp Certified: Terraform Associate}{(10/2021 - 10/2023)}

\cvsection{Achievements}

\cvachievement{\faTrophy}{Best Team}{Delivery best team - 2023}

\divider

\cvachievement{\faTrophy}{Cheer-board}{In Feburary 2024}

\divider

\cvachievement{\faTrophy}{Performer}{TCS}

%% Yeah I didn't spend too much time making all the
%% spacing consistent... sorry. Use \smallskip, \medskip,
%% \bigskip, \vspace etc to make adjustments.
\medskip

\cvsection{Education}

\cvevent{B.E.\ Electronics and Communication}{RGPV}{2012 -- 2016}{}

% \divider

\end{paracol}

\end{document}