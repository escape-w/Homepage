%% Use the "normalphoto" option if you want a normal photo instead of cropped to a circle
% \documentclass[10pt,a4paper,normalphoto]{altacv}

\documentclass[10pt,a4paper,ragged2e,withhyper]{altacv}
%% AltaCV uses the fontawesome5 and packages.
%% See http://texdoc.net/pkg/fontawesome5 for full list of symbols.

% Change the page layout if you need to
\geometry{left=1cm,right=1cm,top=1cm,bottom=1cm,columnsep=1cm}

% The paracol package lets you typeset columns of text in parallel
\usepackage{paracol}

% Change the font if you want to, depending on whether
% you're using pdflatex or xelatex/lualatex
% WHEN COMPILING WITH XELATEX PLEASE USE
% xelatex -shell-escape -output-driver="xdvipdfmx -z 0" sample.tex
\ifxetexorluatex
  % If using xelatex or lualatex:
  \setmainfont{Roboto Slab}
  \setsansfont{Lato}
  \renewcommand{\familydefault}{\sfdefault}
\else
  % If using pdflatex:
  \usepackage[rm]{roboto}
  \usepackage[defaultsans]{lato}
  % \usepackage{sourcesanspro}
  \renewcommand{\familydefault}{\sfdefault}
\fi

% Change the colours if you want to
\definecolor{SlateGrey}{HTML}{2E2E2E}
\definecolor{LightGrey}{HTML}{666666}
\definecolor{DarkPastelRed}{HTML}{450808}
\definecolor{PastelRed}{HTML}{8F0D0D}
\definecolor{GoldenEarth}{HTML}{E7D192}
\colorlet{name}{black}
\colorlet{tagline}{PastelRed}
\colorlet{heading}{DarkPastelRed}
\colorlet{headingrule}{GoldenEarth}
\colorlet{subheading}{PastelRed}
\colorlet{accent}{PastelRed}
\colorlet{emphasis}{SlateGrey}
\colorlet{body}{LightGrey}

% Change some fonts, if necessary
\renewcommand{\namefont}{\Huge\rmfamily\bfseries}
\renewcommand{\personalinfofont}{\footnotesize}
\renewcommand{\cvsectionfont}{\LARGE\rmfamily\bfseries}
\renewcommand{\cvsubsectionfont}{\large\bfseries}


% Change the bullets for itemize and rating marker
% for \cvskill if you want to
\renewcommand{\cvItemMarker}{{\small\textbullet}}
\renewcommand{\cvRatingMarker}{\faCircle}
% ...and the markers for the date/location for \cvevent
% \renewcommand{\cvDateMarker}{\faCalendar*[regular]}
% \renewcommand{\cvLocationMarker}{\faMapMarker*}


% If your CV/résumé is in a language other than English,
% then you probably want to change these so that when you
% copy-paste from the PDF or run pdftotext, the location
% and date marker icons for \cvevent will paste as correct
% translations. For example Spanish:
% \renewcommand{\locationname}{Ubicación}
% \renewcommand{\datename}{Fecha}


%% Use (and optionally edit if necessary) this .tex if you
%% want to use an author-year reference style like APA(6)
%% for your publication list
% % When using APA6 if you need more author names to be listed
% because you're e.g. the 12th author, add apamaxprtauth=12
\usepackage[backend=biber,style=apa6,sorting=ydnt]{biblatex}
\defbibheading{pubtype}{\cvsubsection{#1}}
\renewcommand{\bibsetup}{\vspace*{-\baselineskip}}
\AtEveryBibitem{%
  \makebox[\bibhang][l]{\itemmarker}%
  \iffieldundef{doi}{}{\clearfield{url}}%
}
\setlength{\bibitemsep}{0.25\baselineskip}
\setlength{\bibhang}{1.25em}


%% Use (and optionally edit if necessary) this .tex if you
%% want an originally numerical reference style like IEEE
%% for your publication list
\usepackage[backend=biber,style=ieee,sorting=ydnt,defernumbers=true]{biblatex}
%% For removing numbering entirely when using a numeric style
\setlength{\bibhang}{1.25em}
\DeclareFieldFormat{labelnumberwidth}{\makebox[\bibhang][l]{\itemmarker}}
\setlength{\biblabelsep}{0pt}
\defbibheading{pubtype}{\cvsubsection{#1}}
\renewcommand{\bibsetup}{\vspace*{-\baselineskip}}
\AtEveryBibitem{%
  \iffieldundef{doi}{}{\clearfield{url}}%
}


%% sample.bib contains your publications
\addbibresource{sample.bib}

\begin{document}
\name{Devendra Sahu}
\tagline{Senior DevOps Engineer}
%% You can add multiple photos on the left or right
\photoR{2.8cm}{photo}
% \photoL{2.5cm}{Yacht_High,Suitcase_High}

\personalinfo{%
  % Not all of these are required!
  \email{75devendrasahu@gmail.com}
  \phone{+91-7566503852}
  % \mailaddress{Åddrésş, Street, 00000 Cóuntry}
  \location{Remote, India}
  % \homepage{www.homepage.com}
  % \twitter{@twitterhandle}
  \linkedin{www.linkedin.com/in/thisisdevendrasahu}
  \github{https://github.com/escape-w}
  %\orcid{0000-0000-0000-0000}
  %% You can add your own arbitrary detail with
  %% \printinfo{symbol}{detail}[optional hyperlink prefix]
  % \printinfo{\faPaw}{Hey ho!}[https://example.com/]

  %% Or you can declare your own field with
  %% \NewInfoFiled{fieldname}{symbol}[optional hyperlink prefix] and use it:
  % \NewInfoField{gitlab}{\faGitlab}[https://gitlab.com/]
  % \gitlab{your_id}
  %%
  %% For services and platforms like Mastodon where there isn't a
  %% straightforward relation between the user ID/nickname and the hyperlink,
  %% you can use \printinfo directly e.g.
  % \printinfo{\faMastodon}{@username@instace}[https://instance.url/@username]
  %% But if you absolutely want to create new dedicated info fields for
  %% such platforms, then use \NewInfoField* with a star:
  % \NewInfoField*{mastodon}{\faMastodon}
  %% then you can use \mastodon, with TWO arguments where the 2nd argument is
  %% the full hyperlink.
  % \mastodon{@username@instance}{https://instance.url/@username}
}

\makecvheader
%% Depending on your tastes, you may want to make fonts of itemize environments slightly smaller
% \AtBeginEnvironment{itemize}{\small}

%% Set the left/right column width ratio to 6:4.
\columnratio{0.7}

% Start a 2-column paracol. Both the left and right columns will automatically
% break across pages if things get too long.
\begin{paracol}{2}
\cvsection{Experience}

\cvevent{Senior DevOps Engineer}{Nagarro}{Oct 2021 -- Ongoing}{Remote, Porto}
\begin{itemize}
\item Provisioned and managed Kubernetes products using Terraform and Terragrunt for robust, scalable infrastructure. for Automobile client
\item Migrated Kubernetes application(Artifactory and Sonarqube) deployments from GitHub Actions to Argo CD for better GitOps practices.
\item Migrated Artifactory project onboarding infrastructure code from Terraform to YAML to simplify L2 team operations and reduce support overhead.
\item Built a YAML validation tool to verify custom configuration files for consistency and correctness, Our product have feature to be optin and opt out
\item Developed reusable GitHub Actions workflows at the organization level to optimize and standardize CI/CD pipelines.
\item Created an end-to-end change automation flow for pre-approved/normal change requests using Python, GitHub Actions, and Argo Notifications.
\item Developed containerized AWS Lambda code to auto-generate ServiceNow tickets based on Prometheus alerts.
\item Built and published a Python package to generate Sprint and Operations reports, integrating with Confluence.
\item Implemented a code-based (Terraform) cleanup policy for Artifactory to manage storage and retention automatically.
\item Integrated Dynatrace monitoring events into SonarQube and Artifactory for better observability.
\item Automated GitHub user onboarding through custom GitHub Actions workflows.
\item Developed Azure DevOps pipelines for IBM ACE, API Connect, and Java applications to support end-to-end CI/CD workflows.
\item Reduced manual intervention by automating installation (IBM MQ, ACE) and support tasks across multiple platforms (Linux and Windows)
\item Authored shell scripts to automate system administration and deployment routines.
\end{itemize}
\divider

\cvevent{DevOps Enginee}{Tata Consultancy Services}{Dec 2016 -- Sep 2021}{Gandhinagar, Noida}
\begin{itemize}
\item Engineered and maintained shared Jenkins libraries using Groovy to streamline and standardize CI/CD pipelines across multiple teams.
\item Automated the provisioning of AWS infrastructure using Infrastructure as Code (IaC), improving deployment speed and reducing manual errors.
\item Contributed to an AIOps platform by automating incident resolution workflows in ServiceNow, reducing mean time to resolution (MTTR).
\item Integrated static code analysis and container security tools such as SonarQube and Trivy into CI/CD pipelines to enforce security and quality standards.
\item Managed and automated daily operations for IBM MQ using Shell and PowerShell, improving system reliability and reducing manual effort.
\item Automated SSL certificate issuance and renewal processes for applications using Python and Ansible, enhancing security and compliance.


\end{itemize}

\medskip

%\cvsection{A Day of My Life}

% Adapted from @Jake's answer from http://tex.stackexchange.com/a/82729/226
% \wheelchart{outer radius}{inner radius}{
% comma-separated list of value/text width/color/detail}
%\wheelchart{1.5cm}{0.5cm}{%
%  6/8em/accent!30/{Sleep,\\beautiful sleep},
%  3/8em/accent!40/Hopeful novelist by night,
%  8/8em/accent!60/Daytime job,
%  2/10em/accent/Sports and relaxation,
%  5/6em/accent!20/Spending time with family
%}

% use ONLY \newpage if you want to force a page break for
% ONLY the current column
%\newpage

%\cvsection{Publications}

%% Specify your last name(s) and first name(s) as given in the .bib to automatically bold your own name in the publications list.
%% One caveat: You need to write \bibnamedelima where there's a space in your name for this to work properly; or write \bibnamedelimi if you use initials in the .bib
%% You can specify multiple names, especially if you have changed your name or if you need to highlight multiple authors.
%\mynames{Lim/Lian\bibnamedelima Tze,
%  Wong/Lian\bibnamedelima Tze,
%  Lim/Tracy,
%  Lim/L.\bibnamedelimi T.}
%% MAKE SURE THERE IS NO SPACE AFTER THE FINAL NAME IN YOUR \mynames LIST

%\nocite{*}

%\printbibliography[heading=pubtype,title={\printinfo{\faFile*[regular]}{Articles}},type=article]

%\divider

%\printbibliography[heading=pubtype,title={\printinfo{\faUsers}{Conference Proceedings}},type=inproceedings]

%% Switch to the right column. This will now automatically move to the second
%% page if the content is too long.
\switchcolumn

\cvsection{\small CORE PHILOSOPHY}

\begin{quote}
``Striving for innovation, growth, and excellence in every aspect of life.''
\end{quote}

\cvsection{Skills}
\cvtag{Kubernetes}
\cvtag{AWS}
\cvtag{Terraform}
\cvtag{Ansible} 
\cvtag{Docker}
\cvtag{Github Actions}
\cvtag{Jenkins}
\cvtag{Python}
\cvtag{Azure DevOps}
\cvtag{\small problem-solving}
\cvtag{Containerization}
\cvtag{JIRA}
\cvtag{Automation}
\cvtag{Prometheus}
\cvtag{ArgoCD}
\cvtag{Dynatrace}
\cvtag{ITSM}
\cvtag{SRE}
\cvtag{bash}
\cvtag{scripting}

\cvsection{Certifications}

\cvachievement{\faTrophy}{Certified Kubernetes Administrator}{(04/2023 - 04/2026)}

\divider

\cvachievement{\faTrophy}{Microsoft Certified: DevOps Engineer Expert}{AZ-400}

\divider

\cvachievement{\faTrophy}{HashiCorp Certified: Terraform Associate}{(10/2021 - 10/2023)}

\cvsection{Achievements}

\cvachievement{\faTrophy}{Best Team}{Delivery best team - 2023}

\divider

\cvachievement{\faTrophy}{Cheer-board}{In 2024-25}

\divider

\cvachievement{\faTrophy}{Spot Performer}{TCS}

%% Yeah I didn't spend too much time making all the
%% spacing consistent... sorry. Use \smallskip, \medskip,
%% \bigskip, \vspace etc to make adjustments.
\medskip

\cvsection{Education}

\cvevent{B.E.\ Electronics and Communication}{RGPV}{2012 -- 2016}{}

% \divider

\end{paracol}

\end{document}